\documentclass{beamer}
%\documentclass[notes=only]{beamer}

\usepackage[utf8]{inputenc}
 
\usetheme{Dresden}

\title[Visualisation and GBL for DSA teaching] %optional
{Exploring Visualisation and Game-Based Learning tools for teaching Data Structures and Algorithms}
  
\author{Edward Zhang \and Simon Su}
 
\institute[UoA] % (optional)
{
  Department of ECSE\\
  The University of Auckland
}
 
\date[May 2019] % (optional)
{Literature Review Seminar}

\begin{document}
\frame{\titlepage}
\section{Introduction}
\subsection{Teaching DSA}
\begin{frame}
  \frametitle{Why DSA?}
  Data Structures and Algorithms are an essential topic in Computer Science-related fields, and form the foundation of many higher-level concepts in CS.
  \note{...therefore, it's v. important for students to develop a solid understand of DSA earlier in their studies. This importance is the motivation for focusing our research on developing tools for teaching DSA}
\end{frame}
\begin{frame}
  \frametitle{DSA Curriculum}
  The ACM CS2013 provides guidelines on subjects that should be taught in an undergrad CS course. Algorithms \& Complexity is identified as a core Knowledge Area and within that the knowledge unit of Fundamental Data Structures and Algorithms.
  \pause
  \begin{block}{Our Implementation}
  We will focus on teaching Fundamental DSA for the purposes of the tool we intend to develop.
  \end{block}
\end{frame}
\begin{frame}
  \frametitle{Fundamental DSA in ACM CS2013}
  \begin{itemize}
    \item Simple Numeric Algorithms
    \item Sequential and Binary Search
    \item Quadractic and $\Omega(n\log(n))$ sorting algorithms
    \item Hash tables and collisions
    \item Binary search trees
    \item Graphs and common graph algorithms
    \item Heaps
    \item Pattern matching/string algorithms
  \end{itemize}
\end{frame}
\subsection{Algorithm Visualisation}
\begin{frame}
  \frametitle{Algorithm Visualisation}
  ye
\end{frame}
\subsection{Game-based Learning}
\begin{frame}
  \frametitle{Game-based Learning}
  yeet
\end{frame}
\end{document}