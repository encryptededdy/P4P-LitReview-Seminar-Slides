\documentclass{beamer}
%\documentclass[notes=only]{beamer}

\usepackage[utf8]{inputenc}
 
\usetheme{Dresden}

\title[Visualisation and GBL for DSA teaching] %optional
{Exploring Visualisation and Game-Based Learning tools for teaching Data Structures and Algorithms}
  
\author{Edward Zhang \and Simon Su}
 
\institute[UoA] % (optional)
{
  Department of ECSE\\
  The University of Auckland
}
 
\date[May 2019] % (optional)
{Literature Review Seminar}

\begin{document}
\frame{\titlepage}
\section{Introduction}
\subsection{Research Intent}
\begin{frame}
  \frametitle{Research Intent and Summary}
  What we found...
  \begin{itemize}
    \item Relatively few Game-Based Learning tools for DSA
    \item Game-Based Learning (GBL) tools and Algorithm Visualisations (AVs) are proven tools to improve learning outcomes
    \item Even fewer learning tools contain both GBL and AVs
  \end{itemize}
  \pause
  We propose further research into a tool that offers GBL within a game world, alongside AVs that take advantage of the analogies and interactivity afforded by said game world.
\end{frame}
\subsection{Teaching DSA}
\begin{frame}
  \frametitle{Why DSA?}
  Data Structures and Algorithms are an essential topic in Computer Science-related fields, and form the foundation of many higher-level concepts in CS.
  \note{...therefore, it's v. important for students to develop a solid understand of DSA earlier in their studies. This importance is the motivation for focusing our research on developing tools for teaching DSA}
\end{frame}
\begin{frame}
  \frametitle{DSA Curriculum}
  The ACM CS2013 provides guidelines on subjects that should be taught in an undergrad CS course. Algorithms \& Complexity is identified as a core Knowledge Area and within that the knowledge unit of Fundamental Data Structures and Algorithms.
  \pause
  \begin{block}{Our Implementation}
  We will focus on teaching Fundamental DSA for the purposes of the tool we intend to develop.
  \end{block}
\end{frame}
\begin{frame}
  \frametitle{Fundamental DSA in ACM CS2013}
  \begin{itemize}
    \item Simple Numeric Algorithms
    \item Sequential and Binary Search
    \item Quadractic and $\Omega(n\log(n))$ sorting algorithms
    \item Hash tables and collisions
    \item Binary search trees
    \item Graphs and common graph algorithms
    \item Heaps
    \item Pattern matching/string algorithms
  \end{itemize}
  \note{These are the fundamental DSA as defined by CS2013. While it is beyond the scope of this project to implement all of these in our tool, we will focus on a subset of these DSAs and take into the learning outcomes mentioned in CS2013 when working on the content to implement in our tool}
\end{frame}
\begin{frame}
  \frametitle{DSA Learning Outcomes in CS2013}
  We will target some of the learning outcomes outlined in CS2013...
  \begin{enumerate}
    \item Implement basic numerical algorithms.
    \item Be able to implement common quadratic and O(N log N) sorting algorithms.
    \item Discuss the runtime and memory efficiency of principal algorithms for sorting, searching, and hashing.
    \item Demonstrate the ability to evaluate algorithms, to select from a range of possible options, to provide justification for that selection, and to implement the algorithm in a particular context. 
  \end{enumerate}
  \note{Ideally, our tool would help students achieve these 4 learning outcomes, although the last one may be hard to achieve within a game, further work and research needed.}
\end{frame}
\section{Algorithm Visualisation}
\begin{frame}
  \frametitle{What is Algorithm Visualisation?}
  ye
\end{frame}
\section{Game-based Learning}
\begin{frame}
  \frametitle{What is Game-based Learning?}
  yeet
\end{frame}
\end{document}